%% -*- fill-column: 72; eval: (auto-fill-mode -1); eval: (visual-fill-column-mode 1); eval: (visual-line-mode 1); eval: (adaptive-wrap-prefix-mode 1) -*-
\documentclass[10pt, a4paper, landscape]{article}
\usepackage[T1]{fontenc}
\usepackage{multicol}
\setlength{\columnsep}{5mm}
\usepackage[medium, compact]{titlesec}
\titleformat{\section}[block]{\Large\bfseries\filcenter}{\thesection}{1em}{}
\usepackage{beton}
\DeclareFontSeriesDefault[rm]{bf}{sbc}
\usepackage{eulervm}
\usepackage{amsmath}
\usepackage[margin=0.51in]{geometry}
\usepackage{parskip}
\usepackage{tabularx}
\usepackage{array}
\usepackage{booktabs}
\usepackage{microtype}
%\usepackage{fancyhdr}
%%
%%\usepackage[style=authoryear]{biblatex}
%%\addbibresource{notes.bib}
%%
%%
\newcommand{\defn}[1]{\textbf{#1}}
\newcommand{\isdef}{\stackrel{\text{def}}{=}}
\newcommand{\set}[1]{\mathbold{#1}}
\newcommand{\imag}{\mathrm{i}}
%%
\title{All the maths we know}
\author{}
\date{21 October 2023}
%%
\renewcommand{\arraystretch}{1.2}
\setlength{\tabcolsep}{3pt}
%%
\begin{document}%\maketitle
%%
\begin{multicols*}{3}\raggedcolumns%
%%
\maketitle

\section*{Set theory}
\subsection*{Sets}
\begin{tabularx}{\columnwidth}{@{}l>{\raggedright\arraybackslash}X@{}}
  \toprule
  \defn{Set} & A collection of things, called its \emph{elements} (or \emph{members}). \\

  $x \in X$    & ``$x$ is an element of set $X$.'' \\

  $X\cup Y$     & The \defn{union} of $X$ and $Y$: all elements of $X$ together with all elements of $Y$. \\

  $X\cap Y$     & The \defn{intersection} of $X$ and $Y$; its elements are
  those that are elements of both $X$ and $Y$. \\

  $X \setminus Y$      & The \defn{set difference}: all elements of $X$ that are not elements of $Y$. Sometimes $X-Y$. \\

  $X \subset Y$    & ``Every element of $X$ is an element of $Y$.'' $X$ is a \emph{subset} of~$Y$. \\

  $\{x\in X\mid P(x)\}$ & The set of those elements of $X$ satisfying the
  predicate $P$ (\textit{i.e.}, for which $P(x)$ is true). \\

  $\emptyset$        & The \defn{empty set}---the set with no members. \\

\end{tabularx}

%% ============================================================

\subsection*{Pairs and tuples}
\begin{tabularx}{\columnwidth}{@{}l>{\raggedright\arraybackslash}X@{}}
  \toprule
  \defn{Pair} & An ordered list of two things: like a set but the order matters and they can be the same thing. For example: $(1, 2)$. \\
  
  \defn{$n$-tuple} & An ordered list of $n$ things. For example: $(x_1, x_2,\dotsc, x_n)$. \\

  $X\times Y$      & The \defn{Cartesian product} of $X$ and $Y$ is the set
  of all pairs $(x,y)$ where $x \in X$ and $y \in Y$. \\

  $X^n$ & The set of all $n$-tuples of elements of~$X$. The Cartesian product of $X$ with itself $n$ times. \\

\end{tabularx}

%% ============================================================

\subsection*{Maps}
\begin{tabularx}{\columnwidth}{@{}l>{\raggedright\arraybackslash}X@{}}
  \toprule
  \defn{Map} & A rule which assigns, to every element of a set (called the \emph{domain}), an element of another set (called the \emph{codomain}). Sometimes called a \emph{function}, especially when the codomain is numbers. \\

  $f\colon X\to Y$   & ``$f$ is a map from $X$ to $Y$.'' \\

  $f\colon x \mapsto y$  & ``Specifically, $f$ maps the element $x \in X$ to the
  element $y \in Y$.'' \\
  $f \circ g$    & The map $g$ followed by the map $f$. \\

  \defn{Injection} & A map, $f\colon X \to Y$, is \emph{injective} (also
  \emph{one-to-one}) if no more than one element of $X$ maps to a
  particular element of $Y$. \\

  \defn{Surjection} & A map, $f\colon X \to Y$, is \emph{surjective} (also \emph{onto}) if every element of $Y$ is mapped to by some element of $X$. \\

  \defn{Bijection} & A map, $f\colon X \to Y$, is \emph{bijective} if it
  is both injective and surjective (also “one-to-one and onto”). \\

  $X\cong Y$ & There exists a bijection between sets $X$ and~$Y$. Equivantly: $X$ and $Y$ are \emph{isomorphic as sets}. 
\end{tabularx}


%% ============================================================

\section*{Useful maps}
\begin{tabularx}{\columnwidth}{@{}l>{\raggedright\arraybackslash}X@{}}
  \toprule
  \defn{Operator} & A \emph{binary operator} on a set $\set{X}$ is a map $\star \colon \set{X}\times\set{X} \to \set{X}$. That is, a binary operator takes two elements of $\set{X}$ and returns an element of~$\set{X}$. Operators are typically written in “infix” notation, $x\star y$, rather than in a function notation, $\star(x,y)$. \\

  \defn{Associative} & A binary operator, $\star$, is \emph{associative} if
  \begin{equation*}
    (a \star b) \star c = a \star (b \star c).
  \end{equation*}
  Almost all operators you will meet are associative. Since the order in which the operations are carried out doesn't affect the result, the parentheses are often omitted, as in $a\star b\star c$.\\
  
  \defn{Commutative} & A binary operator, $\star$, is \emph{commutative} if
  \begin{equation*}
    a \star b = b \star a.
  \end{equation*}
\end{tabularx}

%% ============================================================

\section*{Numbers}
\begin{tabularx}{\columnwidth}{@{}l>{\raggedright\arraybackslash}X@{}}
  \toprule
  $\set{N}$ & The \emph{natural numbers:} $0, 1, 2, \dotsc$. \\
  $\set{Z}$ & The \emph{integers:} $\dotsc, -2, -1, 0, 1, 2, \dotsc$. \\
  $\set{Q}$ & The \emph{rationals:} All numbers that can be written
  as $m/n$ where $m$ and $n$ are integers. \\
  $\set{R}$ & The \emph{reals:} The rationals and ``all the numbers in
  between.'' \\
  $\set{R}^+$ & The non-negative reals. \\
  $\set{C}$ & The \emph{complex numbers:} Numbers of the form $a + b\imag$, where $a$ and $b$ are real numbers and $\imag^2=-1$. 
\end{tabularx}

%% ============================================================

\section*{Vector spaces}
\begin{tabularx}{\columnwidth}{@{}l>{\raggedright\arraybackslash}X@{}}
  \toprule \defn{Vector space} & A \emph{real vector space} is a set,
  $V$, together with: (i) a commutative, associative, binary operator,
  $+$, on $V$; (ii) a map, $\cdot\colon \set{R}\times V\to V$; and (iii) a
  distinguished element $\mathbold{0}\in V$, such that:
  \begin{enumerate}
  \item $v + \mathbold{0} = v$ for all $v\in V$;
  \item For any $v\in V$ there exists an element $-v \in V$ such that $v+(-v)=\mathbold{0}$;
  \item $1 \cdot v = v$ for all $v\in V$;
  \item $\alpha\cdot(\beta\cdot v) = (\alpha\beta)\cdot v$ for all $\alpha,\beta\in \set{R}$ and all $v\in V$.
  \item $\alpha\cdot(v+w) = \alpha\cdot v+ \alpha\cdot w$ and $(\alpha+\beta)\cdot v = \alpha\cdot v + \beta \cdot v$ for all $\alpha, \beta\in \set{R}$ and $v,w \in V$.
  \end{enumerate}

  Replacing $\set{R}$ with $\set{C}$ in the above we obtain a
  \emph{complex vector space}. \\

  \defn{Vector} & An element of a vector space. \\

  \defn{Subspace} & A subset of a vector space that is itself a vector
  space with respect to the addition and scalar multiplication inherited
  from the larger space.

  Equivalently: A subset, $U \subset V$ of a vector space, $V$, is a \emph{subspace} if, for all $u, v\in U$ and number $\alpha$, the combination $u+\alpha \cdot v$ is also in~$U$.
\end{tabularx}

\section*{Examples of vector spaces}
\begin{tabularx}{\columnwidth}{@{}l>{\raggedright\arraybackslash}X@{}}
  \toprule
  $\set{R}^n$ (or $\set{C}^n$) & The set of $n$-tuples of $\set{R}$ (or $\set{C}$), together with the operation of “element-wise” addition:
  \begin{multline*}
    (x_1, \dotsc, x_n) + (y_1, \dotsc, y_n) \isdef \\ (x_1+y_1, \dotsc, x_n+y_n) 
  \end{multline*}
  and multiplication by $\set{R}$ (or $\set{C}$):
  \begin{equation*}
    \lambda \cdot (x_1, \dotsc, x_n) \isdef (\lambda x_1, \dotsc, \lambda x_n) 
  \end{equation*} \\
  
  $\set{R}^X$ (or $\set{C}^X$) & For $X$ a set, $\set{R}^X$ denotes the set of all functions $X\to\mathbold{R}$, together with the operation of “pointwise addition”:
  \begin{equation*}
    (f+g)(x) \isdef f(x) + g(x)
  \end{equation*}
  and “pointwise multiplication by $\mathbold{R}$”:
  \begin{equation*}
    (\alpha f)(x) \isdef \alpha f(x).
  \end{equation*}
  (The notation “$(f+g)(x)$” means “the function $f+g$, where $+$ is addition of functions, evaluated at the point $x$.”) \\

  $\mathcal{P}_m(\set{R})$ & The set of polynomials of degree at most $m$. Thus, $f\in\mathcal{P}_m(\set{R})$ implies $f(x) = a_0 + a_1x + \dots + a_m x^m$ for some~$m$. \\

  $\mathcal{P}(\set{R})$ & The set of all polynomials of finite degree.

\end{tabularx}

%% ============================================================

\section*{Combining vector spaces}
\begin{tabularx}{\columnwidth}{@{}l>{\raggedright\arraybackslash}X@{}}
  \toprule
  
  \defn{Sum} & (This definition is not commonplace.) For $U_1, U_2,
  \dotsc, U_n$ subspaces of $V$, their sum is the set of all sums of vectors from the $U_i$:
  \begin{equation*}
    U_1+\dots +U_n \isdef \bigl\{ v_1 + \dots + v_n \bigm\vert v_i \in U_i \bigr\}.
  \end{equation*}
  Equivalently, it is the set of all sums from $\cup_i U_i$ (since sums
  from within the same $U_i$ are already elements of that
  $U_i$). Equivalently, it is the span of $\cup_i U_i$. \\
  
  \defn{Direct sum} & (Version 1: This definition is from Axler.) The
  sum of subspaces, $U_1+\dots + U_n$, is called a \emph{direct sum} if
  the $U_i$ satisfy the following property: if $v_1+\dots +v_n =
  \mathbold{0}$, where $v_i\in U_i$, then each of the $v_i$
  is~$\mathbold{0}$. 

  If the sum of the $U_i$ is a direct sum, it is written $U_1\oplus \dots\oplus U_n$. \\
\end{tabularx}

\section*{Linear independence and bases}
\begin{tabularx}{\columnwidth}{@{}l>{\raggedright\arraybackslash}X@{}}
  \toprule

  \defn{Span} & Let $v_1, \dots, v_n$ be vectors in a real vector space,
  $V$. The \emph{span} of this set is the subspace given by
  \begin{equation*}
    \{ \alpha_1 v_1 +\dots + \alpha_n v_n \mid \alpha_1, \dots \alpha_n\in\set{R} \}
  \end{equation*}
  (and likewise for a complex vector space). \\

  \parbox[t]{0.8in}{\defn{Linear\\ independence}} & Vectors $v_1, \dots, v_n \in \set{R}$ are \emph{linearly independent} if $\alpha_1 v_1 + \dots + \alpha_n v_m = \set{0}$ implies $\alpha_1 = \dots = \alpha_n=0$. \\

  \defn{Basis} (of $V$) & A collection of vectors that (a) spans $V$; (b) is linearly independent. \\

  \defn{Dimension} (of $V$) & The number of elements of any basis of $V$. (Noting that any two bases of $V$ have the same cardinality.) \\
  
\end{tabularx}


%% ============================================================

\section*{Linear maps}
\begin{tabularx}{\columnwidth}{@{}l>{\raggedright\arraybackslash}X@{}}
  \toprule

  \defn{Linear map} & A map, $T\colon V\to W$ (where $V$ and $W$ are
  vector spaces) such that $T(u+v) = T(u)+T(v)$ and $T(\alpha u) = \alpha T(u)$
  for all $u,v\in V$. \\

  $\mathcal{L}(V, W)$ & The set of all linear maps $V\to W$ with the vector space structure given by $(S+\alpha T)(v) \equiv S(v)+\alpha T(v)$ for any $S,T\in V$. \\

  $\mathbf{1}_V$ & The identity map $\mathbf{1}_V\colon V\to V$ where $\mathbf{1}_V\colon v\mapsto v$. \\

  $\circ$ & Composition of linear maps. For linear maps $T\colon V\to W$ and $S\colon W\to X$, their composition $S\circ T$ is that linear map given by $(S\circ T)(v) \equiv S(T(v))$.

  Composition is associative and the identity map is a left and right
  identity. \\

  \defn{Null space} & (Or “kernel”.) Of a linear map, that subspace of the domain that is mapped to the zero vector. \\

  \defn{Range} & Of a linear map, that subspace of the codomain that is mapped to by \emph{some} element of the domain.
  
\end{tabularx}

%% ============================================================

\section*{Matrices}
\begin{tabularx}{\columnwidth}{@{}l>{\raggedright\arraybackslash}X@{}}
  \toprule
  \defn{Matrix} & An $m\times n$ matrix $A$ is $m\times n$ numbers, $A_{ij}$, indexed by $i\in\{1,\dotsc, m\}$ and $j\in\{1,\dotsc, n\}$. (Axler writes the elements as $A_{i,j}$.)

  Conventionally, the matrix is written as a rectangular array of the
  numbers, with $A_{ij}$ being written in the $i$th row (starting at the
  top) and $j$th column (starting from the left).

  (It is usual to index the elements from 1. The only exception I know of is when writing relativitistic tensors.) \\

  $\set{R}^{m,n}$ & The set of all $m\times n$ matrices over $\mathbf{R}$.

  (Everything below can be repeated, replacing “real number” with “complex number.”)

  Addition of matrices and multiplication of matrices by numbers is
  defined elementwise:
  \begin{equation*}
    \begin{gathered}
      (A+B)_{ij} \isdef A_{ij} + B_{ij} \\
      (\lambda A)_{ij} \isdef \lambda A_{ij}.
    \end{gathered}
  \end{equation*}

  Under these definitions, $\set{R}^{m,n}$ is a vector space. (The zero element of this space is the matrix $\boldmath{0}_{ij} = 0$.)
  
  “Multiplication” of matrices is defined between an $l\times m$ and and $m \times n$ matrix, the result being an $l\times n$ matrix:
  \begin{equation*}
    (AB)_{ij} = \sum_{k=1}^m A_{ik}B_{kj}.
  \end{equation*}
  \\
  \defn{Transpose} & Of a matrix \\
\end{tabularx}

\end{multicols*}
\end{document}
