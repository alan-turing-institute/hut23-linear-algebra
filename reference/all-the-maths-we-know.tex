%% -*- fill-column: 72; eval: (auto-fill-mode -1); eval: (visual-fill-column-mode 1); eval: (visual-line-mode 1); eval: (adaptive-wrap-prefix-mode 1) -*-
\documentclass[10pt, a4paper, twocolumn]{article}
%\usepackage{graphicx}
%\usepackage{textcomp}
%\usepackage{amssymb}
%\usepackage{fontspec}
%\usepackage{minted}
\usepackage[T1]{fontenc}
\usepackage{concrete}
\usepackage{eulervm}
\usepackage{amsmath}
%\usepackage[amssymb,sansbold]{concmath}
\usepackage[margin=0.51in]{geometry}
%\setlength{\parskip}{\smallskipamount}
\usepackage{parskip}
\usepackage{tabularx}
\usepackage{array}
\usepackage{booktabs}
%\usepackage{fancyhdr}
%%
%%\usepackage[style=authoryear]{biblatex}
%%\addbibresource{notes.bib}
%%
%%
\newcommand{\defn}[1]{\textbf{\textsf{#1}}}
\newcommand{\set}[1]{\mathbold{#1}}
\newcommand{\imag}{\mathrm{i}}
%%
\title{All the maths we know}
\date{}
\author{}
%%
\renewcommand{\arraystretch}{1.2}
\setlength{\tabcolsep}{3pt}
%%
\begin{document}%\maketitle
\maketitle
\section*{Numbers}
\begin{tabularx}{\columnwidth}{@{}p{0.23\columnwidth}>{\raggedright\arraybackslash}X@{}}
  \toprule
  $\set{N}$ & The \emph{natural numbers:} $0, 1, 2, \dotsc$. \\
  $\set{Z}$ & The \emph{integers:} $\dotsc, -2, -1, 0, 1, 2, \dotsc$. \\
  $\set{Q}$ & The \emph{rationals:} All numbers that can be written
  as $m/n$ where $m$ and $n$ are integers. \\
  $\set{R}$ & The \emph{reals:} The rationals and ``all the numbers in
  between.'' \\
  $\set{R}^+$ & The non-negative reals. \\
  $\set{C}$ & The \emph{complex numbers:} Numbers of the form $a + b\imag$, where $a$ and $b$ are real numbers and $\imag^2=-1$. 
\end{tabularx}

\section*{Set theory}
\subsection*{Sets}
\begin{tabularx}{\columnwidth}{@{}p{0.23\columnwidth}>{\raggedright\arraybackslash}X@{}}
  \toprule
  \defn{Set} & A collection of things, called its \emph{elements} (or \emph{members}). \\

  $x \in X$    & ``$x$ is an element of set $X$.'' \\

  $X\cup Y$     & The \defn{union} of $X$ and $Y$: all elements of $X$ together with all elements of $Y$. \\

  $X\cap Y$     & The \defn{intersection} of $X$ and $Y$; its elements are
  those that are elements of both $X$ and $Y$. \\

  $X \setminus Y$      & The \defn{set difference}: all elements of $X$ that are not elements of $Y$. Sometimes $X-Y$. \\

  $X \subset Y$    & ``Every element of $X$ is an element of $Y$.'' $X$ is a \emph{subset} of~$Y$. \\

  $\{x\in X\mid P(x)\}$ & The set of those elements of $X$ satisfying the
  predicate $P$ (\textit{i.e.}, for which $P(x)$ is true). \\

  $\emptyset$        & The \defn{empty set}---the set with no members. \\

\end{tabularx}

\subsection*{Pairs and tuples}
\begin{tabularx}{\columnwidth}{@{}p{0.23\columnwidth}>{\raggedright\arraybackslash}X@{}}
  \toprule
  \defn{Pair} & An ordered list of two things: like a set but the order matters and they can be the same thing. For example: $(1, 2)$. \\
  
  \defn{$n$-tuple} & An ordered list of $n$ things. For example: $(x_1, x_2,\dotsc, x_n)$. \\

  $X\times Y$      & The \defn{Cartesian product} of $X$ and $Y$ is the set
  of all pairs $(x,y)$ where $x \in X$ and $y \in Y$. \\

  $X^n$ & The set of all $n$-tuples of elements of~$X$. The Cartesian product of $X$ with itself $n$ times. \\

\end{tabularx}

\newpage
\subsection*{Maps}
\begin{tabularx}{\columnwidth}{@{}p{0.23\columnwidth}>{\raggedright\arraybackslash}X@{}}
  \toprule
  \defn{Map} & A rule which assigns, to every element of a set (called the \emph{domain}), an element of another set (called the \emph{codomain} or \emph{range}). Sometimes called a \emph{function}, especially when the codomain is numbers. \\

  $f:X\to Y$   & ``$f$ is a map from $X$ to $Y$.'' \\

  $f:x \mapsto y$  & ``Specifically, $f$ maps the element $x \in X$ to the
  element $y \in Y$.'' \\
  $f \circ g$    & The map $g$ followed by the map $f$. \\

  \defn{Injection} & A map, $f:X \to Y$, is an \emph{injection} (also \emph{one-to-one}) if every element in $X$ is mapped to a unique element of $Y$. \\

  \defn{Surjection} & A map, $f:X \to Y$, is a \emph{surjection} (also \emph{onto}) if every element of $Y$ is mapped to by some element of $X$. \\

  \defn{Bijection} & A map, $f:X \to Y$, is a \emph{bijection} (also an \emph{isomorphism of sets}) if it is one-to-one and onto.   
\end{tabularx}

\section*{Particular spaces}
\begin{tabularx}{\columnwidth}{@{}p{0.23\columnwidth}>{\raggedright\arraybackslash}X@{}}
  \toprule
  $\set{R}^n$ (or $\set{C}^n$) & The set of $n$-tuples of $\set{R}$ (or $\set{C}$) with the operation of “element-wise” addition and multiplication by $\set{R}$ (or $\set{C}$):
  \begin{multline*}
      (x_1, \dotsc, x_n) + (y_1, \dotsc, y_n) := \\ (x_1+y_1, \dotsc, x_n+y_n) 
  \end{multline*}
  and
  \begin{equation*}
    \lambda (x_1, \dotsc, x_n) := (\lambda x_1, \dotsc, \lambda x_n) 
  \end{equation*}
\end{tabularx}


\end{document}
