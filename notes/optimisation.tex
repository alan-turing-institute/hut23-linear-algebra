\documentclass[11pt, a4paper]{article}
\usepackage[utf8]{inputenc}
\usepackage[T1]{fontenc}
\usepackage{beton}
\usepackage{eulervm}
\usepackage{amsmath}
\usepackage{bm}
\DeclareFontSeriesDefault[rm]{bf}{sbc}
% \usepackage{amssymb}
%% Turing grid is 21 columns (of 1cm if we are using A4)
%% Usually 4 "big columns", each of 4 text cols plus 1 gutter col;
%% plus an additional gutter on the left.
\usepackage[left=4cm, textwidth=13cm]{geometry}
\author{James Geddes}
\date{\today}
\title{Optimisation (on vector spaces)}
%%
\DeclareBoldMathCommand{\setR}{R}
\DeclareMathOperator*{\argmin}{arg\,min}
\begin{document}
\maketitle

Here's a classic problem. We are given a real-valued function on some
space $X$, say $f\colon X\to \setR$, and we are to find the point where
$f$ is minimised (if there is one). That is, we are to find
$x_\text{min}$ such that $f(x_\text{min}) < f(x)$ for every other $x\in
X$ that is not~$x_\text{min}$. That is, find
\begin{equation*}
  x_\text{min} = \argmin_{x\in X} f(x).
\end{equation*}


\end{document}
