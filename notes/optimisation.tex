\documentclass[10pt, twocolumn, a4paper]{article}
\usepackage[utf8]{inputenc}
\usepackage[T1]{fontenc}
\usepackage{beton}
\usepackage{eulervm}
\usepackage{amsmath}
\usepackage{bm}
\usepackage{microtype}
\usepackage[left=1cm, right=1cm]{geometry}
\setlength{\columnsep}{1cm}
\usepackage[medium, compact]{titlesec}
\DeclareFontSeriesDefault[rm]{bf}{sbc}
% \usepackage{amssymb}
%% Turing grid is 21 columns (of 1cm if we are using A4)
%% Usually 4 "big columns", each of 4 text cols plus 1 gutter col;
%% plus an additional gutter on the left.
\usepackage[left=4cm, textwidth=13cm]{geometry}
\author{James Geddes}
\date{\today}
\title{Optimisation (on vector spaces)}
%%
\DeclareBoldMathCommand{\setR}{R}
\DeclareMathOperator*{\argmin}{arg\,min}
\begin{document}
\maketitle
\raggedright

Here's a classic problem. We are given a real-valued function on some
space $X$, say $f\colon X\to \setR$, and we are to find the point, $x =
x_\text{min}$, where $f(x)$ attains its minimum value (if there is
one). That is, we are to compute
\begin{equation*}
  x_\text{min} = \argmin_{x\in X} f(x).
\end{equation*}




\end{document}
