\documentclass[10pt, a4paper]{article}
\usepackage[utf8]{inputenc}
\usepackage[T1]{fontenc}
\usepackage{beton}
\usepackage{eulervm}
\usepackage{amsmath}
\usepackage{bm}
\usepackage{microtype}
%\usepackage[medium, compact]{titlesec}
\usepackage[inline]{asymptote}
\usepackage{tikz-cd}
\DeclareFontSeriesDefault[rm]{bf}{sbc}
% \usepackage{amssymb}
%% Turing grid is 21 columns (of 1cm if we are using A4)
%% Usually 4 "big columns", each of 4 text cols plus 1 gutter col;
%% plus an additional gutter on the left.
\usepackage[left=1cm, textwidth=11cm, marginparsep=1cm, marginparwidth=7cm]{geometry}
\usepackage[Ragged, size=footnote, shape=up]{sidenotesplus}
%% We used to use a two-column layout
% \setlength{\columnsep}{1cm}
\title{Simple optimisation (on vector spaces)}
\author{James Geddes}
\date{\today}
%%
\DeclareBoldMathCommand{\setR}{R}
\DeclareBoldMathCommand{\bfC}{C}
\DeclareBoldMathCommand{\bfG}{\Gamma}
\newcommand{\id}{\mathbold{1}} 
\newcommand{\bzero}{\mathbold{0}} % I don't know why \bm{0} fails.
\DeclareMathOperator*{\argmin}{arg\,min}
\DeclareMathOperator{\nullspace}{null}
\newcommand{\eg}{\emph{Example:}}
\newcommand{\ie}{\emph{i.e.}}
\hyphenation{anti-sym-met-ric}
\begin{document}
\maketitle

In this note we derive, without taking derivatives, an expression for
the minimiser of a quadratic form on a finite-dimensional vector
space. We then use this result to solve a related constrained
optimisation problem.

\subsection*{Introduction}
A classic problem is that of finding the location of the minimum of
some real-valued function. That is, given a function,
$f\colon X\to\setR$, defined on some set $X$, we seek $\hat{x}\in X$ such
that $f(\hat{x})<f(x)$ for every other $x\in X$. We call $\hat{x}$ the
\emph{minimiser} of~$f$ and write $\hat{x} = \argmin_{x\in X} f(x)$.

It is a hard problem! It is not just that the function itself might be
complicated in some way; the \emph{domain} of the function might be
complicated: it could be discrete, or high-dimensional, or have some
complicated shape. All of these can prevent an analytic solution and
impede a numerical one.

Sometimes, however, $X$ is a vector space; and then the problem is
much more tractable. In this note we derive closed-form expressions
for the minimisers of certain particularly simple functions on
finite-dimensional vector spaces.

To get a sense for the issues, suppose, for the moment, that we are
interested only in “functions of a single variable;” that is,
functions $f\colon \setR\to\setR$. Here are a few examples where the
minimum can be found without much difficulty.

\eg{} A constant function, $f(x) = a$ attains its minimum value
everywhere. To say it another way, there is no \emph{unique} point at
which it is a minimum.

\eg{} A linear function\sidenote{There is a slight inconsistency in
  terminology here.  A linear \emph{map} from one vector space to
  another necessarily maps zero to zero, whereas a linear
  \emph{function}, like this one, may have a constant term.}
$f(x) = a + bx$ n has no minimum (assuming $b\neq0$). Its value can be
made arbitrarily negative by taking $x$ sufficiently large.

\eg{} A quadratic, $f(x) = a + bx + cx^2$, \emph{may} have a
minimiser. (See figure~\ref{fig:quadratic} for an illustration.)
\begin{marginfigure}
  \begin{center}
    \asyinclude[width=4cm, height=4cm, keepAspect=false]{quadr.asy}
  \end{center}
  \caption{A graph of $f(x) = 9 - 8x + 2x^2$. The minimum occurs at
    $x=2$, as may be seen by “completing the square;” that is, writing
    $f$ as $f(x) = 2{(x-2)}^2+1$.\label{fig:quadratic}}
\end{marginfigure}
Whereas the previous examples could be solved by inspection, for this
example we employ a trick, sometimes known as “completing the square.”
Suppose we can find numbers $\kappa$, $\gamma$, and $\xi$ such that
\begin{equation}
  f(x) = a + bx + cx^2 = \kappa + \gamma{(x - \xi)}^2.
\label{eq:completing-the-square}
\end{equation}
Then, since the addition of a constant, $\kappa$, does not change the
\emph{location} of any mimimum, the minimiser of $f$ is the same the
minimiser of $\gamma{(x-\xi)}^2$. And, since ${(x-\xi)}^2$ is non-negative and
is zero only when $x-\xi=0$, a minimum will exist only when $\gamma >0$ and
that minimum will be at $\hat{x}=\xi$.

The values of $\kappa$, $\gamma$, and $\xi$ can be read off from
eq.~\eqref{eq:completing-the-square} above by equating equal powers
of~$x$: some manipulation gives the minimiser as
$\hat{x}=-\frac{1}{2}c^{-1}b$.

\subsection*{Quadratic forms}
We return, now, to the multidimensional case. Fix, once and for all, a
finite-dimensional, real vector space, $V$, not necessarily
one-dimensional. The example above, of a quadratic function, is
perhaps the simplest possible function on the reals that has a
minimum. What is the equivalent of a ``quadratic function'' on the
vector space~$V$?

Intuitively, we would like to write something of the form
``$f(v) = a+bv+cv^2$.'' The issue is that $f(v)$ is supposed to be a
real number; whereas, on the right-hand side, ``$bv$'' is, on the face
of it, a vector, and ``$cv^2$'' is not even defined. To make sense of
this expression, we must therefore find some way of interpreting the
linear term, ``$bv$,'' and the quadratic term, ``$cv^2$;'' that is, of
forming a number from a vector, and of forming a number from two
vectors, all while “respecting the vector space structure of $V$.”

Recall that the \emph{dual space} of $V$ is the vector space, $V^*$, of
linear maps from $V$ to~$\setR$. That is, $V^*=\mathcal{L}(V, \setR)$. Let
$\tilde{b}\in V^*$ be an element of the dual space. An obvious candidate
for a term linear in $v$ is therefore an expression of the form
$\tilde{b}(v)$.

The quadratic term, ``$cv^2$,'' would seem to involve the ``product of
a vector with itself,'' and that is not an operation that is obviously
available in a general vector space. However, suppose there were a way
to obtain, from some vector $v$, an element $\tilde{v}\in V^*$ of the
dual space. Then the expression $\tilde{v}(v)$ would be a number, which
we might interpret as ``$v^2$.''

Thus, let $\bfC\colon V\to V^*$ be a linear map from $V$ to its
dual. For any vector $v\in V$, we obtain $\bfC(v)\in V^*$ (see
figure~\ref{fig:bilinear-form}). Now suppose $w\in V$ is any other
vector. Since $\bfC(v)$ is a linear map from $V$ to~$\setR$,
the expression $\bigl(\bfC(v)\bigr)(w)$ is a number. We shall think of
this number as the ``product of $v$ and $w$'' (where this “product”
depends of course upon~$\bfC$).
\begin{marginfigure}
  \begin{center}
    \asyinclude[width=5cm]{bilinear-form.asy}
  \end{center}
  \caption{A vector space $V$ and its dual $V^*$, showing: an element
    $v\in V$; a linear map $\bfC\colon V\to V^*$; and the image of $v$ in
    $V^*$ under $\bfC$.\label{fig:bilinear-form}}
\end{marginfigure}

However, the notation is extremely cumbersome. Since we are thinking
of $\bfC$ as ``a way to obtain a number from two vectors,'' we shall
write $\bfC(v,w)$ instead of $\bigl(\bfC(v)\bigr)(w)$. Thus, by
$\bfC(v, w)$ we shall mean, ``apply $\bfC$ to $v$, obtaining an
element, $\bfC(v)\in V^*$, of the dual space, and then apply this
element to $w$, obtaining a number.'' When $\bfC$ is viewed from this
perspective, it is known as a \emph{bilinear form}. In a sense, one
thinks of $\bfC$ as a map from pairs $(v,w)\in V\times V$ to the reals, a map
which is “linear in both $v$ and $w$.”\sidenote{You may recognise this
  construction as equivalent to saying that $\bfC$ is an element of
  ${(V\otimes V)}^*$. Indeed,
  $\mathcal{L}(V\otimes V, \setR) \cong \mathcal{L}\bigl(V, \mathcal{L}(V, \setR)\bigr)$.}

We can now say roughly what is meant by a “quadratic term”: it is an
expression of the form $\bfC(v,v)$ for some bilinear form~$\bfC$.

There is, however, some redundancy in our choice of~$\bfC$. Let
$\bm{A}$ be any bilinear form such that $\bm{A}(v,w)=-\bm{A}(w,v)$ and
consider the bilinear form $\bfC+\bm{A}$. By linearity, we have
$(\bfC+\bm{A})(v, v) = \bfC(v,v)+\bm{A}(v,v)$. However,
$\bm{A}(v,v)=-\bm{A}(v,v)$ (by assumption), whence
$\bm{A}(v,v)=0$. Thus $(\bfC+\bm{A})(v,v)=\bfC(v,v)$; that is,
$\bfC+\bm{A}$ gives rise to the same quadratic form as~$\bfC$.

A bilinear form $\bm{A}$ for which $\bm{A}(v,w)=-\bm{A}(w,v)$ is said
to be \emph{antisymmetric}. Conversely, a bilinear form $\bm{S}$ for
which $\bm{S}(v,w)=\bm{S}(w,v)$ is said to be \emph{symmetric}. From
the foregoing, the addition, to $\bfC$, of any antisymmetric bilinear
form does not change value of $\bfC(v,v)$.

Now consider the identity (for any bilinear form):
\[
  \bfC(v, w) = \frac{1}{2}\bigl[\bfC(v,w) + \bfC(w,v)\bigr]
  + \frac{1}{2}\bigl[\bfC(v,w) - \bfC(w,v)\bigr].
\]
The first term on the right-hand side is symmetric, whereas the second
is antisymmetric. Since the antisymmetric term vanishes when both
arguments are the same, we assume, without loss of generality, that
$\bfC$ is symmetric.\sidenote{The identity
  \[
    \bfC(v, w) = \frac{1}{2}\bfC(v+w, v+w) - \bfC(v,v) -\bfC(w,w),
  \]
  known as the \emph{polarisation identity}, shows that a symmetric
  bilinear form may be recovered from $\bfC(v,v)$.}

We are now in a position to say what we mean by the equivalent of a
quadratric function on a vector space. A \emph{quadratic form on $V$}
is a function of the form 
\begin{equation}
  Q(v) = a - 2\tilde{b}(v) + \bfC(v, v),
  \label{eq:quadratic-function}
\end{equation}
where, on the right-hand side, $a$ is a number, $\tilde{b}$ is an
element of the dual space and $\bfC$ is a symmetric bilinear
form. (The factor of $-2$ is conventional as it simplifies certain
calculations.)

Having written down a function on $V$, we return to the problem of
finding the location of its minimum. That is, what is
\[
\hat{v} = \argmin_{v\in V} Q(v)?
\]
By direct analogy with the one-dimensional case,
eq.~\eqref{eq:completing-the-square}, we might attempt to “complete
the square.” A little work will show that
\begin{equation}
  Q(v) = \bigl[a-\bfC(\xi,\xi)\bigr] + \bfC(v-\xi,v-\xi),
  \label{eq:vector-square}
\end{equation}
where $\xi$ must satisfy
\begin{equation}
  \label{eq:minimiser}
  \bfC(\xi) = \tilde{b}.
\end{equation}

Assume, for the moment, that $\bfC$ is invertible so that
$\xi=\bfC^{-1}(\tilde{b})$ indeed satisfies
eq.~\eqref{eq:minimiser}. This $\xi$ will be a minimiser of $Q$ if
$Q(v)>Q(\xi)$ for all $v \neq \xi$. From eq.~\eqref{eq:vector-square}, that
is equivalent to the condition $\bfC(v-\xi,v-\xi)>\bfC(0,0)$ for all
$v\neq\xi$. However $v$ is arbitrary and so, noting that
$\bfC(0,0)=0$, we obtain
\begin{equation}
  \bfC(v,v) > 0 \quad\text{for all $v\in V$ such that $v\neq0$}.
  \label{eq:positive-definite}
\end{equation}
A symmetric bilinear form for which eq.~\eqref{eq:positive-definite}
holds is said to be \emph{positive definite}. The $\xi$ given by
eq.~\eqref{eq:minimiser}, if it exists, will be a unique minimiser if
and only if $\bfC$ is positive definite.

Now we return to the issue of whether $\bfC$ is invertible. In fact,
it turns out that positive-definiteness implies
invertibility. \emph{Theorem}: Any positive-definite, bilinear form on
a finite-dimensional vector space is invertible. \emph{Proof}: A
linear map on finite-dimensional vector spaces is invertible if it is
injective and surjective. It suffices to show injectivity of
$\bfC\colon V\to V^*$, since the dimension of $V$ is the same as the
dimension of~$V^*$. To show injectivity suppose, for contradiction,
that there is some $u\neq\bzero$ in $V$ such that $\bfC(u)=\bzero$ (with
the right-hand side being the zero element of~$V^*$). Then we would
have $\bfC(u, x) = 0$ for any $x\in V$; whence, in particular,
$\bfC(u,u)=0$, contradicting the assumed positive-definiteness
of~$\bfC$.

All these remarks lead to the following result: Let $V$ be a
finite-dimensional vector space and suppose that $Q(v)$ is a quadratic
form on~$V$ given by $Q(v)=a -2\tilde{b}(v) + \bfC(v,v)$, where
$a$ is a number, $\tilde{b}$ an element of the dual of $V$, and
$\bfC$ a positive-definite, symmetric bilinear form on $V$. Then,
\begin{equation}
  \argmin_{v\in V} Q(v) = \bfC^{-1}(\tilde{b}).
  \label{eq:quadratic-minimiser}
\end{equation}

\subsection*{Constrained optimisation}
Here is a variation on the problem.

Suppose, as before, that $V$ is a finite-dimensional vector space and
$Q(v)$ a quadratric form on~$V$. And suppose, in addition, that we are
given a finite-dimensional vector space, $U$, together with an
injective linear map $\phi:U\to V$. Thus, the composition
$Q\circ\phi$, where $(Q\circ\phi)(u)=Q(\phi(u))$, is a function
on~$U$. The problem is to find the minimiser of this $Q\circ\phi$; that is,
to find $\hat{u}\in U$, where
\begin{equation}
  \hat{u} = \argmin_{u\in U} Q(\phi(u)).
\end{equation}

One way to view this problem is as a constrained minimisation. The
image of $U$ under $\phi$ is a subspace of $V$ (see
figure~\ref{fig:constrained-min}). We might ask for the minimiser of
$Q$ on $V$, subject to the constraint that the minimiser must lie in
the subspace~$\phi[U]$. Having found this minimiser, $\hat{u}$ is then
its pre-image under~$\phi$.
\begin{marginfigure}
  \begin{center}
    \asyinclude[width=5cm]{constrained-min.asy}
  \end{center}
  \caption{The image of $U$ under the linear map $\phi:U\to V$ is a
    subspace, $\phi[U]$, of~$V$. The quadratic form $Q$ is a function on
    $V$, whereas $Q\circ\phi$ is a function
    on~$U$.\label{fig:constrained-min}}
\end{marginfigure}

In fact, however, $Q\circ\phi$ turns out to be itself a quadratic form
on~$U$. If we were able to find an explicit form for $Q\circ\phi$ we would
obtain a minimiser by direct application of
eq.~\eqref{eq:quadratic-minimiser}.

We now show that $Q\circ\phi$ is indeed a quadratic form on~$U$. As before,
write $Q(v)=a -2\tilde{b}(v) + \bfC(v,v)$, where $\tilde{b}\in V^*$ and
$\bfC$ is a symmetric, positive-definite bilinear form. Thus, for any
$u\in U$, we have
$(Q\circ\phi)(u) = a -2\tilde{b}(\phi(u)) + \bfC(\phi(u), \phi(u))$. Our plan is to
find a dual vector $\tilde{\beta}\in U^*$ and a symmetric, positive-definite
bilinear form $\bm{\Gamma}$, such that
$(Q\circ\phi)(u)=a -2\tilde{\beta}(u) + \bm{\Gamma}(u,u)$. Were we to do so, an
application of eq.~\eqref{eq:quadratic-minimiser} would lead to
$\hat{u}=\bm{\Gamma}^{-1}(\tilde{\beta})$.

Consider first the linear term. $\tilde{b}$ is an element of $V^*$ and
we wish to replace it with $\tilde{\beta}$, an element of~$U^*$. Since the
transpose of $\phi$ is a map $\phi^*\colon V^*\to U^*$ (see
figure~\ref{fig:transposes}) one might suspect that a plausible choice
for $\tilde{\beta}$ would be $\tilde{\beta}=\phi^*(\tilde{b})$. And indeed, with
this choice,
$\tilde{\beta}(u) = \phi^*(\tilde{b})(u)= \tilde{b}(\phi(u))$ (by the
definition of the transpose of a linear map).
\begin{marginfigure}
  \[\begin{tikzcd}[column sep=large]
      & U^*  & V^* \arrow[l, "\phi^*"'] \\
      & U \arrow[u, "\bfG"] \arrow[r, "\phi"] & V \arrow[u,
      "\bfC"'] 
    \end{tikzcd}\]
  \caption{The transpose of the map $\phi\colon U\to V$ is the map
    $\phi^*\colon V^*\to U^*$ defined by
    $(\phi^*(\tilde{v}))(u) = \tilde{v}(\phi(u))$ for any $u\in U$ and
    $\tilde{v}\in V^*$. \label{fig:transposes}}
\end{marginfigure} 

Next consider the quadratic term. If the required $\bfG$ exists, it
will be a map $\bfG\colon U\to U^*$. Such a map can be constructed by
setting:
\[
  \bfG = \phi^*\circ \bfC \circ \phi
\]
(see figure~\ref{fig:transposes} again). Note that, for any
$u,w\in U$, we have
$\bfG(u,w) = (\bfG(u))(w) = \phi^*(\bfC(\phi(u)))(w) = \bfC(\phi(u))(\phi(w)) =
\bfC(\phi(u), \phi(w))$. Thus $\bfG(u,u)=\bfC(\phi(u), \phi(u))$ as desired.

We now claim: this $\bfG$, so defined, is a symmetric,
positive-definite bilinear form on~$U$. That it is symmetric follows
from the corresponding property of~$\bfC$, noting that
$\bfG^* = {(\phi^*\circ \bfC \circ \phi)}^* = \phi^*\circ \bfC^* \circ \phi$. That it is positive,
$\bfG(u,u)\geq 0$, follows directly from the corresponding property of
$\bfC$. It remains to show that it is positive-definite; that is, to
show that if $\bfG(u,u)=0$ then necessarily $u=\bzero$. Suppose then that
$u$ is such that $\bfG(u,u)=0$. Since
$\bfG(u,u) = \bfC(\phi(u),\phi(u))$, and since $\bfC$ is positive definite,
it follows that $\phi(u)=\bzero$. But $\phi$ was supposed to be an injective map
and so we must have $u=\bzero$.

In summary, $Q\circ\phi$ is a quadratic form on $U$, of the form
\[
(Q\circ\phi)(u) = a - 2(\phi^*(\tilde{b}))(u) + (\phi^*\circ\bfC\circ\phi)(u, u).
\]
and the minimiser of $Q\circ\phi$ is, therefore, given by
\begin{equation}
\hat{u} = {(\phi^*\circ\bfC\circ\phi)}^{-1}(\phi^*(\tilde{b})).
\label{eq:constrained-minimiser}
\end{equation}

The formula above answers the original question. However, there is
another way of thinking about this answer. Since
$\bfC\circ \bfC^{-1} = \id$, the right-hand side of
eq.~\eqref{eq:constrained-minimiser} may be rewritten as
\[
  \hat{u} = \bigl[{(\phi^*\circ\bfC\circ\phi)}^{-1}\circ\phi^*\circ \bfC\bigr]\circ
  \bfC^{-1}(\tilde{b})
\]
The term $\bfC^{-1}(\tilde{b})$ is the minimiser of $Q$ on~$V$. The
term ${(\phi^*\circ\bfC\circ\phi)}^{-1}\circ\phi^*\circ C$ is a map from
$V$ to $U$; it ``carries the minimiser back to~$U$.'' If $\phi$ were
invertible, this term would reduce to $\phi^{-1}$ and $\hat{u}$ would
just be $\phi^{-1}(\bfC(\tilde{b}))$.  In general, $\phi$ is \emph{not}
invertible and the minimiser of $Q$ on $V$ does not lie in the image
of $U$ under~$\phi$. Nonetheless one sometimes thinks of the larger term
as a \emph{sort} of inverse. 

Write $\psi={(\phi^*\circ\bfC\circ\phi)}^{-1}\circ\phi^*\circ C$. This
$\psi$ satisfies the identity $\psi\circ\phi=\id_U$: it is a left-inverse
of~$\phi$. However, it is not true in general that $\phi\circ\psi$ is the identity
on~$V$; that is, it is not generally true that $\psi$ is a right-inverse
of~$\phi$.

In applications, one typically chooses a basis for $U$ and $V$,
together with the corresponding dual bases for $U^*$ and $V^*$. Often,
the basis for $V$ is chosen to give $\bfC$ the particularly simple
form:
\[
  \bfC(e_i, e_j) =
  \begin{cases}
    1 & \text{if $i=j$;} \\
    0 & \text{otherwise.}
  \end{cases}
\]
When $\bfC$ has this form, the matrix representation of
$\phi^*\circ \bfC$ is the transpose of the matrix representation
of~$\phi$.\sidenote{In general, it is the matrix of $\phi^*$ that is the
  transpose of the matrix of $\phi$.} Thus, interpreting now $\phi$ as a
matrix, we have
\[
  \psi = (\phi^T\phi)^{-1}\phi^T.
\]
In this form, $\psi$ is known as the \emph{Moore--Penrose pseudo-inverse}
of~$\phi$.











\end{document}
