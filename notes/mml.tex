\documentclass[11pt, a4paper]{article}
\usepackage[utf8]{inputenc}
\usepackage[T1]{fontenc}
\usepackage{beton}
\usepackage{eulervm}
\usepackage{amsmath}
\usepackage{bm}
\DeclareFontSeriesDefault[rm]{bf}{sbc}
% \usepackage{amssymb}
%% Turing grid is 21 columns (of 1cm if we are using A4)
%% Usually 4 "big columns", each of 4 text cols plus 1 gutter col;
%% plus an additional gutter on the left.
\usepackage[left=4cm, textwidth=13cm]{geometry}
\author{James Geddes}
\date{\today}
\title{Linear Regression Done Right}
%%
\DeclareBoldMathCommand{\setR}{R}
\begin{document}
\maketitle

This note is an attempt to rewrite chapter 9 of Deisenroth \emph{et
al.}, on linear regression.

\section*{Introduction}

Here are three problems which all seem to have something in common:

\begin{enumerate}
\item An ``urban farm'' grows crops underground in a vacant tunnel. To
  monitor environmental conditions, temperature sensors are placed at
  various locations around the tunnel. The farmers would like a sense
  of the temperature at arbitrary locations.
\item A data scientist is asked to create a model to predict the
  species of a penguin, given some facts about its weight, flipper
  size and bill size.
\item An economist wants to understand whether there is a relationship
  between measures of childhood education and later income.
\end{enumerate}

In each of these examples we believe that there is, in the world, some
sort of map, or function, from an “input” to an “output.” In the first
example, the input is “physical location” and the output is
“temperature;” in the second, the input is “triple of morphological
measurements” and the output is “species;” in the third, the input is
whatever measures of childhood education are used, and the output is,
perhaps, “income at age 30.”

Our challenge is to make a mathematical model of this relationship. In
order to do so, we are typically provided with “data”---a set of
observations of particular inputs and outputs. The mathematical model
of the relationship that we determine is supposed to match this data,
more or less.

On the other hand, there are significant differences between these
problems. In the second and third examples, the map we are seeking
exists somehow \emph{in potentia}, as a relationship between
characteristics of a particular “observational unit” (penguins and
people, respectively). In the first, by contrast, the map just “is” a
thing in the world---the temperature exists at each point in space. The
third example has a “causal” flavour: we presumably hope to be able to
\emph{affect} the output by changing the input. In the second, the
inputs are immutable. Temperature and income take values in a
continuous, totally-ordered set but species comes from a finite,
unordered set. Can we capture the similarities between these different
problems in a way that will let us attempt to solve them?

In each example, there is given a set $X$ (of possible inputs) and a
set $Y$ (of possible outputs) and a collection $i=1,\dots,N$ of pairs
$(x_i, y_i)\in X\times Y$ (the data). Consider the challenge of finding a
map, $\tilde{f}\colon X\to Y$, having the property that $\tilde{f}(x_i)=
y_i$ for each~$i$.

The immediate snag is that finding such a map is \emph{far too easy}:
\begin{equation*}
  \tilde{f}(x) =
  \begin{cases}
    y_i & \text{if $x = x_i$ for some $i$;} \\
      0 & \text{otherwise}.
  \end{cases}
\end{equation*}
Well, that's not what we wanted! It certainly agrees with the data but
since it is zero everywhere else it seems implausible that it
represents the “real world.”

What we meant to ask for was a function that “agrees with the data
\emph{and} is likely to agree with the real function on \emph{other}
values of the input, values we haven't seen yet.” This $\tilde{f}$,
being zero everywhere except just those points where we have data, is
highly unlikely to be a good approximation to the “real” one.

A possible response to this snag is to admit that we don't want
\emph{any} old function, we want one that is in some sense
“reasonable.” What is meant by reasonable presumably depends on the
problem. Reasonableness might take the form of a continuity condition:
we'd like the output to be a continuous function of the input, or
perhaps even smooth. Alternatively, there might be some nomological
reason to prefer a particular kind of function: we might have a theory
of the real function, such as that it satisfies some differential
equation, which our solution also needs to satisfy.

At any rate, suppose that, somehow or other, we have decided on a set
of “reasonable” functions, $\mathcal{F}$, and what we are really asking for is a
function taken from \emph{this} set (and matching the data).

There is now another snag. It turns out to be quite difficult to get
the “size” of $\mathcal{F}$ right. If there are too few functions to draw from,
we run the risk of not finding \emph{any} function that matches the
data.\footnote{In the classical subject of “simple linear regression,”
for example, one is interested in functions $\setR \to \setR$ and one
chooses, for $\mathcal{F}$, all first-order polynomials (that is, straight-line
functions). But hardly any sets of data can be matched by a
first-order polynomial. (Most sets of three points do not lie on a
straight line.)}

In practice, however, we may not want to match the data
\emph{exactly}. There are good reasons for this. One reason is that we
often choose $\mathcal{F}$ to contain only “simple” functions in order to make
the calculations tractable. We might then expect that the “true”
function---the one the real world is using to generate the data---will not
be found within our simplified~$\mathcal{F}$. (For example, in econometrics or
social science, I understand that it's common to fit a linear
relationship, though we've no reason to believe that the real world is
linear.) In this case what we're looking for is a function that is
“close to” the real function and so only approximates the data.

A related reason is that, in the real world, the outputs are not fully
determined by the inputs. (For example, income at age 30 is clearly
not solely determined by education.) If we were able to obtain more
data with inputs $x_i$, we'd likely not get the same $y_i$.\footnote{A
final reason is that the original measurements of $y_i$ might contain
measurement error. Textbooks often harp on this reason, which seems
odd to me: is there \emph{really} so much error in measuring, I don't
know, temperature? I think what's going on is that the other two
reasons are the real reasons but you then have to model them and it's
very hard to justify any particular model. How do you model “we don't
know the real function”? If you assume the measurement error theory,
you get to wave your hands, intone “central limit theorem,” and assume
a normal distribution of the error.











\end{document}

\section*{Notes on the original text}

Like most machine learning techniques, linear regression involves the
representation of a particular real-world problem by mathematical
objects, as well as the use of mathematical methods to solve the
problem. Therefore, in writing an exposition of the subject, one will
move between informal descriptions and formal mathematics; between the
real world and the mathematical world.

It is important that the reader is clear at each point which is which!
Mathematically-inclined texts often conflate informal descriptions and
formal definitions, either by failing to give a proper definition when
one was presaged, or by dropping into heavy mathematics before
clarifying the question. 

For example, Deisenroth \emph{et al.}\ begin as
follows.\footnote{There is one other sentence before this but it
merely introduces the chapter. I've numbered the sentences for ease of
reference; these numbers are not part of the original text. And I have
omitted one other sentence that is not important here.}

\begin{quote}
 [1] In \emph{regression}, we aim to find a function $f$ that maps
 inputs $\bm{x} \in \setR^D$ to corresponding function values $f(\bm{x})
 \in\setR$. [2] We assume that we are given a set of training inputs
 $\bm{x}_n$ and corresponding noisy observations $y_n = f(\bm{x}_n) +
 \epsilon$, where $\epsilon$ is an i.i.d.\ random variable that describes
 measurement/observation noise and potentially unmodeled processes
 [...]. [3] Our task is to find a function that not only models the
 training data, but generalizes well [...].
\end{quote}



\end{document}
