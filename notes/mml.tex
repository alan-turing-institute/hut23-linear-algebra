\documentclass[11pt, a4paper]{article}
\usepackage[utf8]{inputenc}
\usepackage[T1]{fontenc}
\usepackage{concrete}
\usepackage{euler}
\usepackage{amsmath}
% \usepackage{amssymb}
%% Turing grid is 21 columns (of 1cm if we are using A4)
%% Usually 4 "big columns", each of 4 text cols plus 1 gutter col;
%% plus an additional gutter on the left.
\usepackage[verbose, left=5cm, textwidth=8cm]{geometry}
\author{James Geddes}
\date{\today}
\title{Linear Regression Done Right}
\begin{document}

The problem as formulated by Deisenroth \emph{et al.} is roughly as
follows. We are given \(n\) “data points,” that is
\(\mathbf{x}_i\in\mathbf{R}^D\) and \(y_i\in\mathbf{R}\), for
\(i\in\{1,\dotsc,n\}\), and we are to find a function \(f\colon
\mathbf{R}^D \to \mathbf{R}\) such that the \(f(\mathbf{x}_i)\)
approximate the \(y_i\).

The authors describes this notion of approxiation in several
ways. They say that \(f\) should “[model] the training data” and
“generalise well to predicting [values] at input locations that are
not part of the training data.”

Already some things are curious. Why do the \(\mathbf{x}_i\) live in
\(\mathbf{R}^D\), a vector space? It seems likely that we would want to
approximate functions on other spaces. For example, suppose I am given
a temperature, sampled at points on the surface of the earth, and I
wish to find a function that describes the temperature at all
points. Then the approximating function will be \(f\colon
S^2\to\mathbf{R}\). And \(S^2\) is very much not a vector space.
\end{document}
