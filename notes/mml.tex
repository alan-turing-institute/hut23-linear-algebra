\documentclass[10pt, a4paper]{article}
\usepackage[utf8]{inputenc}
\usepackage[T1]{fontenc}
\usepackage{beton}
\usepackage{eulervm}
\usepackage{amsmath}
\usepackage{bm}
\usepackage{microtype}
%\usepackage[medium, compact]{titlesec}
\usepackage[inline]{asymptote}
\DeclareFontSeriesDefault[rm]{bf}{sbc}
% \usepackage{amssymb}
%% Turing grid is 21 columns (of 1cm if we are using A4)
%% Usually 4 "big columns", each of 4 text cols plus 1 gutter col;
%% plus an additional gutter on the left.
\usepackage[left=1cm, textwidth=11cm, marginparsep=1cm, marginparwidth=7cm]{geometry}
\usepackage[Ragged, size=footnote, shape=up]{sidenotesplus}
%% We used to use a two-column layout
% \setlength{\columnsep}{1cm}
\DeclareBoldMathCommand{\setR}{R}
\DeclareMathOperator*{\argmin}{arg\,min}
\newcommand{\eg}{\emph{Example:}}
\newcommand{\ie}{\emph{i.e.}}
\newcommand{\isdef}{\stackrel{\text{def}}{=}}
\hyphenation{anti-sym-met-ric}
%%
\author{James Geddes}
\date{\today}
\title{Linear Regression Done Right}
\begin{document}
\maketitle

This note is an attempt to rewrite chapter 9 of Deisenroth \emph{et
al.}, on linear regression.

\section*{Introduction}

Here are three problems which all seem to have something in common:

\begin{enumerate}
\item An ``urban farm'' grows crops underground in a vacant tunnel. To
  monitor environmental conditions, temperature sensors are placed at
  various locations around the tunnel. The farmers would like a sense
  of the temperature at arbitrary locations.
\item A data scientist is asked to create a model to predict the
  species of a penguin, given some facts about its weight, flipper
  size and bill size.
\item An economist wants to understand whether there is a relationship
  between measures of childhood education and later income.
\end{enumerate}

In each of these examples one might imagine that there is, in the
world, some sort of map, or function, from an “input” to an “output.”
In the underground farm, the input is “physical location” and the
output is “temperature;” for the data scientist, the input is “triple
of morphological measurements” and the output is “species;” for the
economist, the input is whatever measures of childhood education are
used, and the output is, perhaps, “income at age 30.”

In each case, there is given a set of observations, known as “the
data,” comprising particular inputs and their corresponding
outputs. The challenge is somehow to estimate, or approximate, or
model, the “true” relationship between inputs and outputs, so that the
model of the relationship matches the data, more or
less.\sidenote|-3in|{Why “more or less”? Why not match the data
  exactly? One reason, often cited, is “measurement error:” the idea
  that the $y_i$s are not measured exactly but contain some “noise”
  and so will differ from those generated by the real map. (Textbooks
  often harp on this reason, which seems odd to me: is there
  \emph{really} so much error in measuring, I don't know,
  temperature?)\sidepar%
  %
  A related but perhaps more plausible reason is that, in the real
  world, the outputs are not likely to be fully determined by the
  measured inputs. For example, income at age 30 is clearly not
  determined by education alone: multiple other inputs, many of which
  are difficult to measure, must play a role. If we were somehow able
  to obtain another measurement of $y$ for the very same input $x_i$,
  these other, hidden, inputs would presumably be different and we
  would not obtain the same~$y_i$.\sidepar%
  %
  Finally, it has been historical practice to consider only “simple”
  models, in order to make the calculations tractable. We might then
  expect that the “true” function---the one the real world is using to
  generate the data---will not be matched by our simplified model. For
  example, in econometrics or social science, it's not uncommon to fit
  a linear relationship, even when there is no reason to believe that
  the real world is linear. In this case what we're looking for is a
  function that is “close to” the real function and so only
  approximates the data.}

There are differences between these three problems and one might
wonder how significant they are. Can we capture the similarities
between these different problems in a way that will let us attempt to
find useful models?

To be more specific, suppose there is given a set, $X$, of possible
inputs, a set, $Y$, of possible outputs, and a collection of $d$ pairs
$(x_i, y_i)\in X\times Y$ (for $i=1,\dots,d$), that comprise the data. For
now, we suppose no extra structure on $X$; however, on $Y$ we imagine
that there is some notion of “closeness” (to be made precise
later). Consider the challenge of finding a map,
$\hat{f}\colon X\to Y$, having the property that $\hat{f}(x_i)$ is close
to $y_i$ for each~$i$. (All of this is somewhat informal. In
particular, the notion of closeness might differ by data point.)

One immediate snag is that finding such a map is \emph{far too
  easy}. Consider:
\begin{equation*}
  \hat{f}(x) =
  \begin{cases}
    y_i & \text{if $x = x_i$ for some $i$;} \\
      0 & \text{otherwise}.
  \end{cases}
\end{equation*}
This map is not just “close” to the data: it exactly matches the
data. However, since it is zero everywhere else, it seems implausible
that it represents the real world. What we presumably meant to ask for
was a function which not only agrees with the data but is also likely
to agree with the real function on \emph{other} values of the input,
values we haven't seen yet.

A possible response to this snag is to observe that this function,
$\hat{f}$, is somehow “physically unreasonable.” We just don't expect
the real world to behave as $\hat{f}$ does. On this view, what we
should do is identify a set of “reasonable” functions and then search
for $\hat{f}$ only within this set. In order to identify the set of
reasonable functions, one might appeal to a nomological condition: if
we had a \emph{theory} of the world, one which requires the function
to satisfy some differential equation, say, we might demand that our
approximating function also satisifies this law.\sidenote{More
  commonly, one tends to conflate reasonableness with pragmatism: one
  allows only functions that are “simple,” in some way, in order to
  make the problem tractable. “Simple” might mean “smooth,” or
  “low-order” or even “linear.”}

Whatever one's definition of “reasonable,” imagine that, somehow or
other, there is fixed a particular collection of functions,
$\mathcal{F}$. What is really wanted is a function taken from \emph{this}
collection that is in some way “close” to the data.\sidenote{It turns
  out to be quite difficult to get the “size” of $\mathcal{F}$ just right. If
  there are too few functions to draw from, we run the risk of not
  being able to match the real function; if there are too many, we run
  the risk of choosing one that is not physically reasonable just
  because it is a good match to the data.} In order to make any
further progress it is now necessary to say something more specific
about the meaning of “close.”

\section{Least squares}

A popular version of “close” runs as follows. First, suppose that the
set of “possible outputs,” $Y$, is the real
numbers,~$\setR$.\sidenote{On the face of it, this is quite a strong
  supposition. For example, it is not true of any of the examples at
  the beginning of this note.} There is, on the reals, an obvious
notion of the closeness of two numbers such as $y_i$ and $f(x_i)$;
namely the value of $\lvert f(x_i)-y_i\rvert$. In the version of
function fitting known as \emph{least squares}, the closeness of a
function $f$ to the whole dataset $(x_i, y_i)$ is measured by the
value of the expression
\begin{equation}
  \label{eq:least-squares-loss}
  L(f) = \sum_{i=1}^d {\bigl(f(x_i)-y_i\bigr)}^2  
\end{equation}
That is, one says that the function $f$ is close to the data just in
case the value of $L(f)$ is small. More generally, a function like $L$
is known as the \emph{loss function} (or sometimes “loss functional,”
to indicate that it depends on a function, although a functional is
also a function) and this particular one is sometimes called the
“quadratic loss” or ”squared loss.”

The above definition of close has some nice properties: it is related
to the distance of each $y_i$ from the $f(x_i)$; it is non-negative;
and it is zero only when the function exactly matches the data. One
sometimes attempts to \emph{justify} this definition, often by appeal
to some statistical principles. However, it is also true that this
particular choice of $L(f)$ can be written in a way that is suggestive
of a further simplification and which makes the problem significantly
more tractable. 

\begin{marginfigure}
  \begin{center}
    \asyinclude[width=4cm, height=4cm, keepAspect=false]{evalmap.asy}
  \end{center}
  \caption{The evaluation map, $\mathcal{E}_{\bm{x}}$, acts on a function
    $f\in\mathcal{F}$ to produce a point in~$\setR^d$. The “loss function”
    measures the distance from this point to the data, 
    $\bm{y}$.\label{fig:evalmap-on-f}}
\end{marginfigure}
Recall that $\setR^d$ is the vector space of length-$d$ tuples of
reals, with addition of tuples “element-wise.” Thus, one element
of~$\setR^d$ is the tuple $\bm{y}=(y_1,\dotsc,y_d)$. Another element
of $\setR^d$ is given by the following map:
\[
  \begin{aligned}
    \mathcal{E}_{\bm{x}} \colon \mathcal{F} &\to \setR^d \\
    f &\mapsto (f(x_1), \dotsc, f(x_d)).
  \end{aligned}
\]
The subscript $\bm{x}$ (in $\mathcal{E}_{\bm{x}}$) is there as a reminder that
the map depends upon the data. For fixed data, $\mathcal{E}_{\bm{x}}$ maps a
function, $f$, to the element of $\setR^d$ given by
$(f(x_1), \dotsc, f(x_d))$: this map is known as the \emph{evaluation
  map}. (See figure~\ref{fig:evalmap-on-f} for an illustratation.)

The idea, now, is to express $L(f)$ as the “distance” between
$\mathcal{E}_{\bm{x}}(f)$ and $\bm{y}$; or, equivalently, as the “length” of
$\mathcal{E}_{\bm{x}}(f)-\bm{y}$. A natural notion of length in
$\setR^d$ is the Euclidean distance: for
$\bm{v}=(v_1, v_2, \dotsc, v_d)$, the “length” of $\bm{v}$ is given by
${\lVert \bm{v} \rVert}^2 = \sum_{i=1}^d v_i^2$. The expression
$\lVert\cdot\rVert$ is called the \emph{canonical norm}.  Making use of
this notation, eq.~\eqref{eq:least-squares-loss} may be rewritten as
\begin{equation}
  \label{eq:norm-loss}
  L(f) = {\Vert \mathcal{E}_{\bm{x}}(f) - \bm{y}\rVert }^2.
\end{equation}

We now summarise the discussion to this point. Our problem was to
choose, from a set of functions, $\mathcal{F}$, a particular function,
$\hat{f}$, which should approximate given data. The sense in which we
mean “approximates” is that the values of the function, evaluated at
the $x$-values of the data, should be “close to” the $y$-values of the
data. And the notion of “close to” that we have assumed that of
“having a small Euclidean distance in the space $\setR^d$. In brief,
we are to solve the following minimisation problem:
\begin{equation}
  \label{eq:least-squares}
  \hat{f} = \argmin_{f\in\mathcal{F}} {\lVert \mathcal{E}_{\bm{x}}(f) - \bm{y}\rVert}^2,
\end{equation}
where, in this minimisation, the data are held fixed.

Can we say anything about solutions to this problem? One possibility
is that the evaluation map,
$\mathcal{E}_{\bm{x}}\colon \mathcal{F} \to \setR^d$, is surjective. In that case, there is
at least one function, $\hat{f}$, which reproduces the data exactly
and therefore solves eq.~\ref{eq:least-squares}; namely,
$\hat{f}=\mathcal{E}_{\bm{x}}^{-1}(\bm{y})$. But in practice this is not the
usual case (at least, not for this loss function). A surjective
$\mathcal{E}_{\bm{x}}$ typically means that $\mathcal{F}$ is “too large” and we are at
risk of choosing unreasonable functions which just happen to match
these particular data.

The case when $\mathcal{E}_{\bm{x}}$ is not surjective is illustrated in
figure~\ref{fig:evalmap-in}.
\begin{marginfigure}
  \begin{center}
    \asyinclude[width=4cm, height=4cm, keepAspect=false]{evalmap-in.asy}
  \end{center}
  \caption{The image of $\mathcal{F}$ under $\mathcal{E}_{\bm{x}}$ is a subset of
    $\setR^d$.\label{fig:evalmap-in}}
\end{marginfigure}
Under $\mathcal{E}_{\bm{x}}$, the space of functions is mapped to a subset
of~$\setR^d$. To solve eq.~\eqref{eq:least-squares} one might imagine
finding the point in this subspace that is closest to $\bm{y}$ and
then finding the the preimage of this point (in the sense of
${\lVert\cdot\rVert}^2$) in~$\mathcal{F}$. Alternatively, one might consider
“distance to $\bm{y}$” as a function on
$\mathcal{E}_{\bm{x}}[\mathcal{F}]$, pull back this function to
$\mathcal{F}$, and then find the minimum there. In general, no closed-form
solution is available under either approach. However, it turns out
that there is a certain class of problems for which a closed-form
solution \emph{can} be found, and that is when $\mathcal{F}$ is itself a vector
space.

\section{Linear regression}



$\mathcal{F}$ is a vector space!

So $\mathcal{E}_{\bm{x}}$ is a linear map!

And the pullback of $L(f)$ under $\mathcal{E}_{\bm{x}}$ is a quadratic form.

Which we can mimise,






\end{document}

However, note that $\bm{x}$ is \emph{not}, in general, a vector,
because $X$ is not, in general, a vector space. One is perfectly
entitled to write, say, $\bm{x}=(x_1, \dotsc, x_d)$, but what is
denoted is a tuple, not a vector.


To make this connection clearer, we introduce on
$\setR^d$ a bilinear form, $\Delta$, as follows. For vectors
$\bm{u} = (u_1,\dotsc,u_d)$ and $\bm{v}=(v_1,\dotsc,v_d)$, set
\[
  \Delta(\bm{u}, \bm{v}) = \sum_{i=1}^d u_i v_i.
\]
It is immediate that $\Delta$ is symmetric and positive definite. To see
the latter, note that for $\bm{v}\in\setR^d$, the value of
$\Delta(\bm{v}, \bm{v})$ is the square of the Euclidean length of
$\bm{v}$, as computed by Pythogoras' rule. In fact, applying $\Delta$ to a
single vector (twice) is such a common use of $\Delta$ that one often sees
an abbreviated notion that removes the need to \emph{write} the vector
twice. We write:
\[
  {\lVert \bm{v} \rVert}_\Delta^2 \equiv \Delta(\bm{v}, \bm{v}).
\]
(The subscript $\Delta$ on ${\Vert\bm{v}\rVert}_\Delta^2$ is there because there are
sometimes other kinds of distance.)

Finally, using this accumulated notation, we can write
\begin{equation}
  L(f) = \mathcal{E}_{\bm{x}}(f) - \bm{y}.
\end{equation}


\section*{Notes on the original text}

Like most machine learning techniques, linear regression involves the
representation of a particular real-world problem by mathematical
objects, as well as the use of mathematical methods to solve the
problem. Therefore, in writing an exposition of the subject, one will
move between informal descriptions and formal mathematics; between the
real world and the mathematical world.

It is important that the reader is clear at each point which is which!
Mathematically-inclined texts often conflate informal descriptions and
formal definitions, either by failing to give a proper definition when
one was presaged, or by dropping into heavy mathematics before
clarifying the question. 

For example, Deisenroth \emph{et al.}\ begin as
follows.\sidenote{There is one other sentence before this but it
merely introduces the chapter. I've numbered the sentences for ease of
reference; these numbers are not part of the original text. And I have
omitted one other sentence that is not important here.}

\begin{quote}
 [1] In \emph{regression}, we aim to find a function $f$ that maps
 inputs $\bm{x} \in \setR^D$ to corresponding function values $f(\bm{x})
 \in\setR$. [2] We assume that we are given a set of training inputs
 $\bm{x}_n$ and corresponding noisy observations $y_n = f(\bm{x}_n) +
 \epsilon$, where $\epsilon$ is an i.i.d.\ random variable that describes
 measurement/observation noise and potentially unmodeled processes
 [\ldots]. [3] Our task is to find a function that not only models the
 training data, but generalizes well [\ldots].
\end{quote}




