%% -*- fill-column: 72; eval: (auto-fill-mode -1); eval: (visual-fill-column-mode 1); eval: (visual-line-mode 1); eval: (adaptive-wrap-prefix-mode 1) -*-
\documentclass[10pt, a4paper, twocolumn]{article}
\usepackage[T1]{fontenc}
\usepackage{concrete}
\usepackage{eulervm}
\usepackage{amssymb}
\usepackage{amsmath}
\usepackage[margin=0.51in]{geometry}
\usepackage{parskip}
\usepackage{tabularx}
\usepackage{array}
\usepackage{booktabs}
\newcommand{\defn}[1]{\textbf{\textsf{#1}}}
\newcommand{\set}[1]{\mathbold{#1}}
\newcommand{\ii}{\mathrm{i}}
%%
\title{Axler 1.A: Selected Exercises}
\date{}
\author{}
%%
\begin{document}\maketitle
\section*{Ex.~1}
\begin{quote}
Suppose $a$ and $b$ are real numbers, not both~0. Find real numbers $c$ and $d$ such that
$1 / (a + b\ii) = c + d\ii$.
\end{quote}
A complex number consists of a “real part” and an ``imaginary part.'' The condition says that $a + b\ii$ is a complex number whose real part is $a$ and whose imaginary part is $b$ (and that this number is not zero).

What the question is really asking is,``how do you compute the multiplicative inverse of a complex number?''

If we knew $c$ and $d$ we would know the complex number that is the multiplicative inverse of $a + b\ii$. It \emph{looks} like we’ve got two unknowns ($c$ and $d$) and only one equation. That’s not good! In general, one needs two equations to find two unknowns: one equation is under-determined.

But in fact, every equation involving complex numbers is really two equations in disguise: One obtained by equating the real parts, and one obtained by equating the imaginary parts. So the trick here is to get this into a form where we can equate the real and imaginary parts of both sides.

Multiplying both sides by $a + b\ii$ gives:
\begin{equation*}
  1 = (a + b\ii) (c + d\ii)
\end{equation*}

Expanding out the right hand side and collecting terms gives:
\begin{equation*}
  1 = (ac - bd) + (bc + ad) \ii
\end{equation*}
(The minus sign is there because $\ii^2 = -1$.) 

Now it’s in the right form: we can “read off” the real parts and the imaginary parts of both sides of this equation. Since the real parts have to be equal:
\begin{equation*}
  1 = (ac - bd);
\end{equation*}
and since the imaginary parts have to be equal:
\begin{equation*}
0 = (bc + ad).
\end{equation*}

That’s two equations and there are two unknowns ($c$ and $d$) so now we’re in with a fighting chance. One way to proceed is to use the second equation to get $d$ in terms of $c$:
\begin{equation*}
  d = -bc / a.
\end{equation*}
Then substitute for $d$ in the first equation leaving something that can be solved for $c$.

The results are:
\begin{equation*}
  c = \frac{a}{a^2 + b^2}
  \quad\text{and}\quad 
    d = \frac{-b}{a^2 + b^2}.
\end{equation*}

\section*{Ex.~4}
\begin{quote}
Show that $\alpha + \beta = \beta + \alpha$ for all $\alpha, \beta \in \set{C}$.
\end{quote}
The technical term for an operation where the order doesn't matter is a \emph{commutative} operation.

One's first thought on reading this question is likely to be something like, ``isn't that just true?'' However, the thing to note is that, since $\alpha$ and $\beta$ are complex numbers, the `$+$' operation is ``complex addition,'' not ordinary addition of real numbers. It \emph{is} obvious that ordinary addition is commutative (or, at least, we all believe that fact) whereas the claim that complex addition is commutative is something that requires a demonstration.

The definition of addition of complex numbers is given on page~2 of
Axler; it is: ``add together separately the real parts and the imaginary
parts.'' Put like that, the result follows immediately: adding together
the real parts is commutative (because it is ordinary addition) and the
same is true of adding together the imaginary parts. And so the addition
of complex numbers does not depend on the order either.

\subsection*{A wrinkle}

Well, perhaps the result follows immediately but it's worth giving the
argument again in symbols because it is surprising how easy it is to switch unknowingly from one meaning to another.

Here's Axler's definition of complex addition. The sum of $a+b\ii$ and $c+d\ii$ is given by:
\begin{equation*}
  (a+b\ii) + (c +d\ii) = (a+c) + (b+d)\ii.
\end{equation*}
That looks a lot like the thing it is supposed to represent



In fact, let's put a dot over the `$+$' to indicate that this is complex addition (this notation is not standard!).

So what we're being asked to show is that:
\begin{equation*}
  \alpha \dotplus \beta = \beta \dotplus \alpha \quad\text{(where $\dotplus$ is complex addition)}.
\end{equation*}
That makes it look more like there's a thing to show because it's not obvious how $\dotplus$ behaves.

That operation is defined on page~2 of Axler and is roughly, ``add together separately the real parts and the imaginary parts.'' More carefully:
\begin{equation*}
  (a + b\ii) + (c + d\ii) = (a+c) + (b+d)\ii.
\end{equation*}




\end{document}
